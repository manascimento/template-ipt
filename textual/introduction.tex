Some basic ways to manipulate text are \textit{italics} and \textbf{bold}. One can reference Figures (see Figure \ref{fig:ipt} for an example) as well as cite references, which are defined in the \textit{references.bib} file.\parencite{Rosa2020ModeloServicos}

The \textit{Bibliography}\index{bibliography}, \textit{List of Figures} and \textit{List of Tables} are all automatically generated and references will be updated automatically as well. This means that if you've defined a citation but are not referencing it, it will not appear in the \textit{Bibliography}. This also means that any Figure / Table / Citations numbers are automatically updated as well. Numbering is done by order-of-appearance.

One can create an itemized list:
\begin{itemize}
    \item item a
    \item item b
    \item ...
\end{itemize}

Or enumerate them:
\begin{enumerate}
    \item item x
    \item item y
    \item ...
\end{enumerate}

\begin{figure}[h]
  \centering
  \caption{\textit{An image of the IPT logo.}}
  \label{fig:ipt}
  \includegraphics[width=.5\textwidth]{figures/ipt.jpg}
  \objectsource{\objectsourceauthor}
\end{figure}

A table with three columns can be seen in Table \ref{board:requirements}.
\begin{table}[H]
  \centering
  \board{A simple quadro}
  \label{board:requirements}
  \begin{tabularx}{\textwidth} { 
    | >{\raggedright\arraybackslash}X 
    | >{\raggedright\arraybackslash}X 
    | >{\raggedright\arraybackslash}X | }
   \hline
   \textbf{Nr} & \textbf{Req} & \textbf{Weight} \\
   \hline
    1 & Price & High \\
   \hline
    2 & Support & Middle \\
   \hline
    3 & Variety & Low \\
  \hline
  \end{tabularx}
  \objectsource{\objectsourceauthor}
\end{table}
