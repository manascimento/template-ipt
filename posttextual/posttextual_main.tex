% Bibliography
\printbibliography[heading=bibintoc,title={REFERÊNCIAS}]
\thispagestyle{myheadings} %Change Page number location
\pagebreak

% Complementary bibliography
\begin{refsection}[./references/unusedreferences.bib]
  \nocite{*}
  \printbibliography[heading=bibintoc,title={BIBLIOGRAFIA COMPLEMENTAR}]
  \thispagestyle{myheadings} %Change Page number location
  \pagebreak
\end{refsection}

% Glossary
\printglossary[title={GLOSSÁRIO},toctitle=GLOSSÁRIO,style=listgroup,nonumberlist]
\pagebreak

% Appendix
\appendix
\begin{appendices}

% Change the title formatting only for the appendix
\titleformat{\chapter}{\normalfont\normalsize\bfseries}{\chaptertitlename{} \thechapter\space--}{.5em}{\MakeUppercase}

% Add to Table of Content (TOC) 
\addtocontents{toc} {
  \protect\renewcommand{\protect\cftchapaftersnum}{\space\textendash\space}
  \addtolength{\cftchapnumwidth}{3pt}
}

% Chapters
\chapter{PRIMEIRO} \label{app:appendixa} \lipsum[1-2]

\pagebreak

\chapter{SEGUNDO} \label{app:appendixb}
\lipsum[1-2]

\pagebreak
\end{appendices}

% Annex
% The strategy here was to use the appendices environment variable to create the annex section.
\begin{appendices}

% Change the title to annex
\renewcommand{\appendixname}{ANEXO}

% Restart the counter for annex
\setcounter{chapter}{0}

% Change the title formatting only for the annex
\titleformat{\chapter}{\normalfont\normalsize\bfseries}{\chaptertitlename{} \thechapter\space--}{.5em}{\MakeUppercase}

% Chapters
\chapter{PRIMEIRO} \label{ann:annexa}
\lipsum[1-2]

\pagebreak

\chapter{SEGUNDO} \label{ann:annexb}
\lipsum[1-2]

\pagebreak
\end{appendices}

% Index
\printindex
\thispagestyle{myheadings} % Change Page number location
